\documentclass[a4paper,man,natbib]{apa6}

\usepackage[english]{babel}
% \usepackage[utf8x]{inputenc}
\usepackage{amsmath}
\usepackage{graphicx}
\usepackage[colorinlistoftodos]{todonotes}
\usepackage[hyphens,spaces,obeyspaces]{url}
\urlstyle{same}


\title{A Disjunctive Interpretation of the Mathematical Aesthetic}
\shorttitle{A Disjunctive Mathematical Aesthetic}
\author{Braden Webb}
\affiliation{Department of Philosophy, Brigham Young University\\
PHIL 495R: Directed Readings in the Philosophy of Mathematics\\
Dr. Derek Haderlie\\
22 April 2023}

\abstract{Although mathematicians have long used aesthetic language to describe their work,
characterizing theorems and proofs as ``beautiful'' or ``elegant'', it is only in recent decades 
that a substantial literature on mathematical aesthetics has emerged in the philosophical community. 
In this paper, I argue for a disjunctive view of the relationship between aesthetics and epistemics in
mathematics.}

\begin{document}
\maketitle

\section{Introduction}
Carl Friedrich Gauss described mathematics as the ``queen of the sciences'', and its apparent purity and 
seemingly \textit{a priori} nature has led many to consider it the epitome of what can be achieved through
the application of logic and human reason. Because of these uniquely rigorous and epistemically sound
foundations, the field has often been upheld as a valuable case study for philosophers \citep{shapiro_thinking_2000},
demanding explanation for how knowledge of abstract mathematical objects and structures is possible. 
Yet despite the general emphasis on the epistemic properties of mathematics, working mathematicians have 
also been long known to use aesthetic language to describe their work, characterizing theorems and proofs as
``beautiful'' or ``elegant''. 

For example, \cite{hardy_mathematicians_1940} writes that
\begin{quotation}
      The mathematician’s patterns, like the painter’s or the poet’s, must be beautiful; the ideas, like the 
      colours or the words, must fit together in a harmonious way. Beauty is the first test: there is no 
      permanent place in the world for ugly mathematics.
\end{quotation}
He also argues that the aesthetic value of a mathematical theorem depends on many properties, including
beauty, seriousness, generality, depth, unexpectedness, inevitability, and 
economy \citep{hardy_mathematicians_1940}. A recent empirical survey \citep{johnson_intuitions_2019} finds that mathematicians
tend to consistently describe the aesthetic properties of mathematical arguments along dimensions of
elegance, profundity, and clarity. Indeed, Aristotle himself claimed that ``the main species of beauty are orderly 
arrangement, proportion, and definiteness'', and that ``these are especially manifested by the mathematical sciences''
\cite[VIII, 1078a]{aristotle_metaphysics_nodate}.

In recent decades, a substantial literature on the mathematical aesthetic has emerged. Several open questions have
featured prominently in the discussion, including: `What is the nature of mathematical beauty?' 
\citep{blasjo_definition_2012,cellucci_mathematical_2015}, `Is aesthetic value essential to mathematical structures
and arguments, or projected onto them by observers?' \citep{mcallister_mathematical_2005}, and `What is the relationship
between aesthetic value and epistemic value in mathematics?' \citep{todd_unmasking_2008,todd_fitting_2018}. Although
I will discuss each of these questions as necessary to provide context, I am particularly interested in the 
aesthetic-epistemic dichotomy, and will focus on elucidating the distinction in this paper. In particular, I will argue for a
disjunctive view of the relationship between aesthetics and epistemics in mathematics, in which the two...

\subsection{A Note on Empirical Observation}

In considering the study of the philosophy of mathematics, \cite{shapiro_thinking_2000} characterizes
the view that philosophy ``precedes'' or ``determines'' practice as the \textit{philosophy-first} paradigm.
For example, those who subscribe to the view that modern mathematical discourse is incoherent due to unsound
metaphysical or logical presuppositions, and that the field should therefore be revised to conform to
first principles determined by philosophy, would be considered to be working within the philosophy-first
paradigm. Many (if not most) philosophers, and certainly most mathematicians, reject this paradigm. For the
purposes of this paper, I also reject this view and instead hope to provide a coherent, post-hoc account
of the aesthetic experience of mathematicians. 

One interesting consequence of this approach is that empirical observations of mathematicians' aesthetic experiences
become not only relevant to the discussion, but in many cases necessary. This is especially so when considering the 
nature of the mathematical aesthetic, and how it relates to aesthetic experiences in other domains. Indeed, as
\cite{inglis_beauty_2015} point out, the Wittgensteinian take on understanding the meaning of the aesthetic lexicon 
used in mathematics requires us to inquire into how those words are actually used by mathematicians. As a result, 
much of the essential empirical research on this topic comes not only from those interested in mathematics and aesthetics, 
but also mathematics education \citep{dreyfus_aesthetics_1986,inglis_beauty_2015,larvor_diversity_2016,inglis_are_2020,
sa_mathematicians_2023}. I will refer to several of these important papers throughout this essay.

\section{Types of Aesthetic Judgments}

\cite{sinclair_roles_2004,sinclair_aesthetic_2011} characterizes three different roles that aesthetic judgments play in mathematics:
\textit{evaluative}, \textit{generative}, and \textit{motivational}. Evaluative judgments are those that concern
the significance or elegance of particular mathematical result. These are the types of judgments made by publishers and
reviewers of journals in mathematics in deeming whether or not particular results or theorems are ``worthy'' of 
publication, and the judgments according to which mathematicians determine how best to present their work for 
others to read. Evaluative judgments are perhaps the most easily recognizable
of the different types of judgments, as they are the ones for which aesthetic language is most often used and discussed.
Generative judgments, on the otherhand, are often made implicitly by mathematicians in the process of developing theories
and attempting to prove new results---they concern which strategies should be used and which avenues should be pursued
to most effectively solve a problem or provide a proof. Finally, motivational judgments are those that determine which
problems are most worthy of study in the first place, and which areas of mathematics should be explored.


\section{The Aesthetic-Epistemic Dichotomy}

\subsection{Beautiful and False}

\begin{itemize}
      \item The Beauty of the Langlands Program?? (actually this would be a great example for motivational, but not
            evaluative)
      \item The Beauty of the Hilbert Programme
      \item The Beauty 
\end{itemize}

\section{Authenticity of the Mathematical Aesthetic}

Several people have argued that the aesthetic experience of mathematicians is, in fact, identical with the epistemic
 \{cite people!\}.
\cite{todd_unmasking_2008} is perhaps the most prominent example of this view. He first claims that the vocabulary used to
describe mathematical beauty is much weaker than that used to define beauty in other domains. A scientist, engineer,
or data analyst may exclaim ``Beatiful!'' upon discovering a result or successfully processing a dataset, but this
excitement (and perhaps even intellectual pleasure) does not indicate the appreciation of beauty. The argument is that
a similar situation occurs in mathematics, and the paucity of expressive power of the aesthetic vocabulary in mathematics
is due to the fact that the words used are not even expressing aesthetic judgments, but some other type of judgment. The
other argument that \cite{todd_unmasking_2008} makes is that, when forced to explain their evaluation of an argument
as aesthetically valuable, the criteria that mathematicians fall back on are almost always epistemically valuable as well.
If all of the elements of an elegant happen to coincide with those of most valid proofs, what evidence do we have to say
that there is a distinction between the two types of value?



and the surveys
conducted by \cite{inglis_beauty_2015} of contemporary research mathematicians will be used to support several
of my arguments. 

While the results of \cite{inglis_beauty_2015} are fascinating, it is important to note that their empirical
semantics do not compare the way in which mathematicians use aesthetic language to the way in which non-mathematicians
use aesthetic language. As a result, while they seem to have shown that individual mathematicians do use aesthetic
language to describe mathematical objects and procedures in consistent ways---in particular, that there are
at least four different dimensions along which they classify these mathematical structures---, they have not shown that this use
of language is either consistent or inconsistent with the use of aesthetic language in other domains. 


I want to argue that a disjunctive view of the relationship between aesthetics and epistemics in mathematics
and science is entirely plausible, and provides a sufficient explanation for the use of aesthetic language
in mathematics. This contradicts the view espoused by \cite{todd_unmasking_2008}, who argues that the
use of aesthetic language in mathematics is entirely epistemic in nature. As particular counterexamples,
I will use the instances in science of past models which were, and could still now be, considered beautiful (such
as the Ptolemaic model of the solar system or the notion of a pervasive, light-conducting ether permeating the
universe) and within mathematics, the entire Hilbert programme as an instance of a beautiful, but false theory.
Since we also have plenty of examples of ``ugly'' truths in both mathematics and science, this seems mostly sufficient
to demonstrate the validity of the disjunctive model of aesthetic experience in mathematics and science.

However, I'm pretty sure there have to be some sort of flaws in that logic, since it seems so simple and easy. I know
I can also draw on the work from \cite{inglis_beauty_2015} to support my conclusions, but I think there are deeper
reasons for which \cite{todd_unmasking_2008} rejects the disjunctive model. I think I need to address those reasons,
but I'm not currently in a sufficiently awake state to do so. I'll try to get back to this tomorrow.

\begin{itemize}
      \item Introduction
      \item Types of Aesthetic Judgments
      \item The Aesthetic-Epistemic Dichotomy

      \item The Authenticity of the Mathematical Aesthetic---that these judgments (at least the evaluative ones) are
      genuinely aesthetic in nature, at least in the same way that aesthetic judgments in other domains are
      \item 
\end{itemize}


\bibliography{example}

\end{document}

%
% Please see the package documentation for more information
% on the APA6 document class:
%
% http://www.ctan.org/pkg/apa6
%