\documentclass[a4paper,man,natbib]{apa6}

\usepackage[english]{babel}
% \usepackage[utf8x]{inputenc}
\usepackage{amsmath}
\usepackage{graphicx}
\usepackage[colorinlistoftodos]{todonotes}
\usepackage[hyphens,spaces,obeyspaces]{url}
\urlstyle{same}


\title{A Disjunctive Interpretation of the Mathematical Aesthetic}
\shorttitle{A Disjunctive Mathematical Aesthetic}
\author{Braden Webb}
\affiliation{Department of Philosophy, Brigham Young University\\
PHIL 495R: Directed Readings in the Philosophy of Mathematics\\
Dr. Derek Haderlie\\
22 April 2023}

\abstract{Although mathematicians have long used aesthetic language to describe their work,
characterizing theorems and proofs as ``beautiful'' or ``elegant'', it is only in recent decades 
that a substantial literature on mathematical aesthetics has emerged in the philosophical community. 
In this paper, I argue for a disjunctive view of the relationship between aesthetics and epistemics in
mathematics.}

\begin{document}
\maketitle

\section{Introduction}
Carl Friedrich Gauss described mathematics as the ``queen of the sciences'', and its apparent purity and 
\textit{a priori} nature has led many to consider it the epitome of what can be achieved through
the application of logic and human reason. Because of these uniquely rigorous and epistemologically sound
foundations, the field has often been upheld as a valuable case study for philosophers \citep{shapiro_thinking_2000},
demanding explanation for how knowledge of abstract mathematical objects and structures is possible. 
Yet despite the general emphasis on the epistemic properties of mathematics, working mathematicians have 
also been long known to use aesthetic language to describe their work, characterizing theorems and proofs as
``beautiful'' or ``elegant''. 

For example, \cite{hardy_mathematicians_1940} writes that
\begin{quotation}
      The mathematician’s patterns, like the painter’s or the poet’s, must be beautiful; the ideas, like the 
      colours or the words, must fit together in a harmonious way. Beauty is the first test: there is no 
      permanent place in the world for ugly mathematics.
\end{quotation}
He also argues that the aesthetic value of a mathematical theorem depends on many properties, including
beauty, seriousness, generality, depth, unexpectedness, inevitability, and 
economy \citep{hardy_mathematicians_1940}. His list, however, is certainly not comprehensive. A recent empirical survey
\citep{johnson_intuitions_2019} finds that mathematicians tend to consistently describe the aesthetic properties of 
mathematical arguments along dimensions of elegance, profundity, and clarity. Indeed, Aristotle himself 
claimed that ``the main species of beauty are orderly arrangement, proportion, and definiteness'', 
and that ``these are especially manifested by the mathematical sciences''
\cite[VIII, 1078a]{aristotle_metaphysics_nodate}.

In recent decades, a substantial literature on the mathematical aesthetic has emerged. Several open questions have
featured prominently in the discussion, including: `What is the nature of mathematical beauty?' 
\citep{blasjo_definition_2012,cellucci_mathematical_2015}, `Is aesthetic value essential to mathematical structures
and arguments, or projected onto them by observers?' \citep{mcallister_mathematical_2005}, and `What is the relationship
between aesthetic value and epistemic value in mathematics?' \citep{todd_unmasking_2008,todd_fitting_2018}. Although
I will discuss each of these questions as necessary to provide context, I am particularly interested in the 
aesthetic-epistemic dichotomy, and will focus on elucidating the distinction in this paper. In particular, I will argue for a
disjunctive view of the relationship between aesthetics and epistemics in mathematics, in which neither beauty
nor truth is a necessary or sufficient condition for the other. Before doing so, I will first discuss a framework
for understanding the different types of aesthetic judgments that mathematicians make, and then consider the current
challenges to the disjunctive view. Finally, I will argue that the disjunctive view is the most plausible model
for the apparent coincidence of the standards for aesthetic and epistemic value in mathematics, despite leaving
many open questions about the relationship between the two.

\subsection{A Note on Empirical Observation}

In considering the study of the philosophy of mathematics, \cite{shapiro_thinking_2000} characterizes
the view that philosophy ``precedes'' or ``determines'' practice as the \textit{philosophy-first} paradigm.
For example, those who subscribe to the view that modern mathematical discourse is incoherent due to unsound
metaphysical or logical presuppositions, and that the field should therefore be revised to conform to
first principles determined by philosophy, would be considered to be working within the philosophy-first
paradigm. Many philosophers (if not most), and certainly most mathematicians, reject this paradigm. For the
purposes of this paper, I also reject this view and instead hope to provide a coherent, post-hoc account
of the aesthetic experience of mathematicians. 

One interesting consequence of this approach is that empirical observations of mathematicians' aesthetic experiences
become not only relevant to the discussion, but in many cases, necessary. This is especially so when considering the 
nature of the mathematical aesthetic, and how it relates to aesthetic experiences in other domains. Indeed, if we accept the
Wittgensteinian perspective on semantics, we must inquire into how the aesthetic lexicon is actually used by mathematicians if we hope to 
understand the meaning that they are trying to convey. As a result, much of the essential empirical research on this topic comes not only 
from those interested in mathematics and aesthetics, but also mathematics education \citep{dreyfus_aesthetics_1986,
inglis_beauty_2015,larvor_diversity_2016,inglis_are_2020,sa_mathematicians_2023}. I will refer to several of these 
important papers throughout this essay.

\section{Types of Aesthetic Judgments}
Although it is tempting to treat the aesthetic experience of mathematics as a monolithic phenomenon, mathematicians in
fact make several types of aesthetic judgments in various different contexts. 
\cite{sinclair_roles_2004,sinclair_aesthetic_2011} characterizes three different roles that aesthetic judgments play in mathematics:
\textit{evaluative}, \textit{generative}, and \textit{motivational}. Evaluative judgments are those that concern
the significance or elegance of particular mathematical result. These are the types of judgments made by publishers and
reviewers of journals in mathematics in deeming whether or not particular results or theorems are worthy of 
publication, and the judgments according to which mathematicians determine how best to present their work for 
others to read. Evaluative judgments are perhaps the most easily recognizable
of the different types of judgments, as they are the ones for which aesthetic language is most often used and discussed.
Generative judgments, on the otherhand, are often made implicitly by mathematicians in the process of developing theories
and attempting to prove new results---they concern which strategies should be used and which avenues should be pursued
to most effectively solve a problem or provide a proof. Finally, motivational judgments are those that determine which
problems are most worthy of study in the first place, and which areas of mathematics should be explored.

As \cite{ivanova_aesthetic_2017} points out, there is a similar feel to the distinction between these types of judgments 
and the distinction drawn by Reichenbach \citep{sep-reichenbach} between the \textit{context of discovery} and the
\textit{context of justification} in the philosophy of science. In the traditional interpretation of these two contexts,
the former occurs when a mathematician or scientist is exploring the space of possible explanations and theories (i.e., 
making motivational judgments), while the latter occurs when they are attempting to justify the truth of a particular 
theory or explanation (i.e., making genertative judgments while developing a proof). Although the context of justification 
is generally considered to be a strictly rational process (and in mathematics, logical consistency certainly reigns above all other
considerations), \cite{cellucci_mathematical_2015} argues that aesthetic generative judgments also play a role 
in this context. Even this process, however, primarily entails selecting from among possible options---only at the
lower level of lemmas and detailed arguments rather than on the level of theory as a whole.

There now seems to be a general consensus that the types of judgments which we have traditionally deemed `aesthetic' judments
and which use a subset of the same aesthetic vocabulary with which we are familiar, truly do occur in both the context
of discovery and the context of justification, as well as the evaluative judgments made after-the-fact. The question of 
whether or not these judgments are truly aesthetic in nature is a separate one, and one which I will address in the next
section. 

% This threefold distinction provides a valuable framework for understanding the role of aesthetics in mathematics,
% and I will refer back to it throughout this paper, but it is certainly not complete. For example, it is unclear
% whether or not 

% The questions of whether or not
% the aesthetic qualities of these judgments are inherent to their objects or projected onto them by observers 
% (as argued by \cite{rota_phenomenology_1997}) remain unresolved, and 

\section{Authenticity of the Mathematical Aesthetic}

The concept of the aesthetic has always been notoriously difficult to define, and has been linked to the concept
of taste since the days of Hume \citep{sep-aesthetic-concept}. For many early theorists, aesthetic
judgments were taken to be defined by immediacy and disinterest. By being immediate, it was meant that judgments
of beauty were not based on reason or rational deliberation, but were instead as direct and instinctive as
sense perception. By being disinterested, it was meant that aesthetic judgments were not based on any particular
desire or interest, but were instead made for their own sake. Hume in particular argued that although aesthetic
judgments, though immediate, were based on general principles about the world, an improved capacity for making
these judgments (also known as taste) could be cultivated through experience. This simplified picture of the
early aesthetic tradition is, of course, rather controversial in reality, but does provide some important
bakcground for discussion in this paper.

In the realm of mathematics, the Humean position would be that as people develop expertise in the field, 
their capacity for making aesthetic judgments about mathematical objects and arguments improves 
as they develop a more refined mathematical taste. Indeed, this does appear to reflect reality in that
most mathematicians relate to perceiving a result as elegant or beautiful, while nine out of twelve of the
non-mathematicians in a survey conducted by \cite{zeki_experience_2014} indicated that they do not
``experience an emotional response'' when they encounter ``beautiful equations''. 

Another difficult question in aesthetics is whether or not aesthetic judgments are objective or subjective.
Objective aesthetic judgments are those that hold universally, independent of the observer, while subjective
aesthetic judgments are those that vary from observer to observer. A notion of inter-subjective agreement
is also often invoked, in which aesthetic judgments tend to be shared by a particular group of society. It is
with this conception of inter-subjective agreement that \cite{mcallister_mathematical_2005} put forth one
of the strongest explanations of the relationship between aesthetic and epistemic value in mathematics. He
presents a theory of ``aesthetic induction'', in which the aesthetic value that the mathematical (or scientific)
community place on a particular result or theory is a result of the state of the community's current ``aesthetic
canon'', and the degree to which the result accords with that canon. For example, the results of denying the 
parallel postulate were thought to be ugly and ``repugnant'' by the mathematical community up until the 19th 
century, when such non-Euclidean geometries were shown to be consistent models of the other axioms of Euclidean
geometry. Since that time, the aesthetic canon of the mathematical community has shifted to deemphasize graphical
representations and visual intuition, and to instead emphasize logical coherence and consistency. This shift
in the aesthetic canon has resulted in the acceptance of non-Euclidean geometries as very beautiful and elegant.

McAllister actually spends more time discussing the ``aesthetic induction'' in scientific theories more than
in mathematics (giving quantum mechanics as a particular case study), but the analogy holds quite well. In 
essence, he argues that the aesthetic canon of a society
at a given time is ultimately determined by the empirical success of existing theories, and the properties such
theories are known to have. For example, since symmetry and parsimony are known to be properties of many 
successful theories, they are considered to be beautiful---but only for as long as they continue to be consistently 
associated with successful theories. This theory holds up well in the case of one recent development in 
mathematics---the rise of the computer-assisted proof, particularly to prove the four-color theorem. Since at no
time in history had proofs of mathematical theorems been provided which could not, at least in theory, be checked 
by holding the entire argument in one's head, the idea of a proof that could not be verified by a human was
disconcerting to many mathematicians. Indeed, \cite{rota_phenomenology_1997} explains that
\begin{quotation}
      Mathematicians have been ambivalent about such a verification. On the one hand, every mathematician 
      professes to be satisfied to learn that the conjecture has been settled. On the other hand, the 
      behavior of the community of mathematicians belies such a feeling of satisfaction.
\end{quotation}
This is a perfect example of the aesthetic canon of the mathematical community being challenged by a new
development in the field. The computer-assisted proof of the four-color theorem was not, and continues to not be 
accepted as beautiful by the community, but McAllister predicts that computer-assisted proofs can become more 
beautiful overtime as the standards of the community change due to the inter-subjective nature of the 
aesthetic \cite{mcallister_mathematical_2005}.

However, McAllister's position leads to some of its own problems.

Several people have argued that the aesthetic experience of mathematicians is, in fact, identical with the epistemic
 \{cite people!\}.
\cite{todd_unmasking_2008} is perhaps the most prominent example of someone sympathetic to this view. While he
does fully take this position himself, he provides strong grounds for suspecting that the distinction between
the aesthetic and epistemic in mathematics is, in fact, illusory, and therefore argues that we should assume 
this is the case until we can demonstrate otherwise. He first claims that the vocabulary used to
describe mathematical beauty is much weaker than that used to define beauty in other domains. A scientist, engineer,
or data analyst may exclaim ``Beatiful!'' upon making a discovery or successfully processing a dataset, but we generally
interpret this exclamation as expressing only excitement, intellectual pleasure, or satisfaction rather than appreciation
of beauty. The argument is that a similar situation occurs in mathematics, and explaining why the the paucity of expressive
power of the aesthetic vocabulary in mathematics
is due to the fact that the words used are not even expressing aesthetic judgments, but some other type of judgment. The
second claim is that, when forced to explain their evaluation of an argument
as aesthetically valuable, the criteria that mathematicians fall back on are almost always epistemically valuable as well.
If all of the elements of \textit{elegant} proofs happen to coincide with those of \textit{valid} proofs, what evidence do we have to say
that there is a distinction between the two types of value?

I find the first claim rather unconvincing. While it is true that the range of adjectives used by mathematicians to describe
the aesthetic qualities of their work is limited, this is not by itself evidence that they are not having an affective
experience. Indeed, in comparison to the size and scope of the art world (which forms the obvious basis of comparison),
the mathematical community is both much smaller and much less accessible. Mathematicians create and define new terms
constantly as they develop new mathematical structures and theories, but while most of humanity has taken time to 
appreciate paintings, music, and literature, only a small fraction of the population has the opportunity to appreciate
mathematical proofs, and its therefore not surprising that the vocabulary used to describe the aesthetic experience
of mathematics is less developed.

The second claim is more nuanced and is worth considering in more detail. While it is possible that a more thorough 
empirical investigation into the reasons mathematicians give for describing proofs as beautiful could reveal instances
where the criteria for aesthetic and epistemic value diverge, I do not here object to the current empirical claim 
that the criteria often coincide. Instead, I would like to further investigate the implications of this coincidence
to determine whether or not this precludes a distinction between the two types of value. In particular, I
would like to consider whether or not the putative conflation of the aesthetic and epistemic occurs in the same
way for evaluative, generative, and motivational judgments, or if there are differences between the three types.


\section{The Aesthetic-Epistemic Dichotomy}

I would also argue that the very fact that mathematicians occasionally disagree about the aesthetic value of a proof
is evidence for a disjunctive view.

\subsection{Beautiful and False}

\begin{itemize}
      \item The Beauty of the Langlands Program?? (actually this would be a great example for motivational, but not
            evaluative)
      \item The Beauty of the Hilbert Programme
      \item The Beauty 
\end{itemize}


There are a number of possible explanations for
the apparent coincidence of the standards for aesthetic and epistemic value in mathematics. The first is that the

While the results of \cite{inglis_beauty_2015} are fascinating, it is important to note that their empirical
semantics do not compare the way in which mathematicians use aesthetic language to the way in which non-mathematicians
use aesthetic language. As a result, while they seem to have shown that individual mathematicians do use aesthetic
language to describe mathematical objects and procedures in consistent ways---in particular, that there are
at least four different dimensions along which they classify these mathematical structures---, they have not shown that this use
of language is either consistent or inconsistent with the use of aesthetic discourse in other domains. 

I want to argue that a disjunctive view of the relationship between aesthetics and epistemics in mathematics
and science is entirely plausible, and provides a sufficient explanation for the use of aesthetic language
in mathematics. This contradicts the view espoused by \cite{todd_unmasking_2008}, who argues that the
use of aesthetic language in mathematics is entirely epistemic in nature. As particular counterexamples,
I will use the instances in science of past models which were, and could still now be, considered beautiful (such
as the Ptolemaic model of the solar system or the notion of a pervasive, light-conducting ether permeating the
universe) and within mathematics, the entire Hilbert programme as an instance of a beautiful, but false theory.
Since we also have plenty of examples of ``ugly'' truths in both mathematics and science, this seems mostly sufficient
to demonstrate the validity of the disjunctive model of aesthetic experience in mathematics and science.

However, I'm pretty sure there have to be some sort of flaws in that logic, since it seems so simple and easy. I know
I can also draw on the work from \cite{inglis_beauty_2015} to support my conclusions, but I think there are deeper
reasons for which \cite{todd_unmasking_2008} rejects the disjunctive model. I think I need to address those reasons,
but I'm not currently in a sufficiently awake state to do so. I'll try to get back to this tomorrow.

\begin{itemize}
      \item Introduction
      \item Types of Aesthetic Judgments
      \item The Aesthetic-Epistemic Dichotomy

      \item The Authenticity of the Mathematical Aesthetic---that these judgments (at least the evaluative ones) are
      genuinely aesthetic in nature, at least in the same way that aesthetic judgments in other domains are
      \item 
\end{itemize}


\bibliography{example}

\end{document}

%
% Please see the package documentation for more information
% on the APA6 document class:
%
% http://www.ctan.org/pkg/apa6
%