\documentclass[a4paper,man,natbib]{apa6}

\usepackage[english]{babel}
% \usepackage[utf8x]{inputenc}
\usepackage{amsmath}
\usepackage{graphicx}
\usepackage[colorinlistoftodos]{todonotes}
\usepackage[hyphens,spaces,obeyspaces]{url}
\urlstyle{same}


\title{A Disjunctive Interpretation of the Mathematical Aesthetic}
\shorttitle{A Disjunctive Mathematical Aesthetic}
\author{Braden Webb}
\affiliation{Department of Philosophy, Brigham Young University\\
PHIL 495R: Directed Readings in the Philosophy of Mathematics\\
Dr. Derek Haderlie\\
22 April 2023}

\abstract{Although mathematicians have long used aesthetic language to describe their work,
characterizing theorems and proofs as ``beautiful'' or ``elegant'', it is only in recent decades 
that a substantial literature on mathematical aesthetics has emerged in the philosophical community. 
In this paper, I argue that the aesthetic experience of mathematics cannot be reduced to epistemic
considerations, provide examples of beautiful-yet-false theories in mathematics, and conclude that 
as a result, only a disjunctive view of the relationship between aesthetics and epistemics in mathematics,
in which neither beauty nor truth is a necessary or sufficient condition for the other, is plausible. This 
is done while remaining agnostic as to the subjectivity or objectivity of mathematical beauty.}

\begin{document}
\maketitle

\section{Introduction}
Carl Friedrich Gauss described mathematics as the ``queen of the sciences'', and its apparent purity and 
\textit{a priori} nature has led many to consider it the epitome of what can be achieved through
the application of logic and human reason. Because of these uniquely rigorous and epistemologically sound
foundations, the field has often been upheld as a valuable case study for philosophers \citep{shapiro_thinking_2000},
demanding explanation for how knowledge of abstract mathematical objects and structures is possible. 
Yet despite the general emphasis on the epistemic properties of mathematics, working mathematicians have 
also been long known to use aesthetic language to describe their work, characterizing theorems and proofs as
``beautiful'' or ``elegant''. 

For example, \cite{hardy_mathematicians_1940} writes that
\begin{quotation}
      The mathematician’s patterns, like the painter’s or the poet’s, must be beautiful; the ideas, like the 
      colours or the words, must fit together in a harmonious way. Beauty is the first test: there is no 
      permanent place in the world for ugly mathematics.
\end{quotation}
He also argues that the aesthetic value of a mathematical theorem depends on many properties, including
beauty, seriousness, generality, depth, unexpectedness, inevitability, and 
economy \citep{hardy_mathematicians_1940}. His list, however, is certainly not comprehensive. A recent empirical survey
\citep{johnson_intuitions_2019} finds that mathematicians tend to consistently describe the aesthetic properties of 
mathematical arguments along dimensions of elegance, profundity, and clarity; moreover, Aristotle himself 
claimed that ``the main species of beauty are orderly arrangement, proportion, and definiteness'', 
and that ``these are especially manifested by the mathematical sciences''
\cite[VIII, 1078a]{aristotle_metaphysics_nodate}.

In recent decades, a substantial literature on the mathematical aesthetic has emerged. Several open questions have
featured prominently in the discussion, including: `What is the nature of mathematical beauty?' 
\citep{blasjo_definition_2012,cellucci_mathematical_2015}, `Is aesthetic value essential to mathematical structures
and arguments, or projected onto them by observers?' \citep{mcallister_mathematical_2005}, and `What is the relationship
between aesthetic value and epistemic value in mathematics?' \citep{todd_unmasking_2008,todd_fitting_2018}. Although
I will discuss each of these questions as necessary to provide context, I am particularly interested in the 
aesthetic-epistemic dichotomy, and will focus on elucidating the distinction in this paper. In particular, I will argue for a
disjunctive view of the relationship between aesthetics and epistemics in mathematics, in which neither beauty
nor truth is a necessary or sufficient condition for the other. Before doing so, I will first discuss a framework
for understanding the different types of aesthetic judgments that mathematicians make, and then consider the current
challenges to the disjunctive view. Finally, I will argue that the disjunctive view is the most plausible model
for the apparent coincidence of the standards for aesthetic and epistemic value in mathematics, despite leaving
many open questions about the relationship between the two.

\subsection{A Note on Empirical Observation}

In considering the study of the philosophy of mathematics, \cite{shapiro_thinking_2000} characterizes
the view that philosophy ``precedes'' or ``determines'' practice as the \textit{philosophy-first} paradigm.
For example, those who subscribe to the view that modern mathematical discourse is incoherent due to unsound
metaphysical or logical presuppositions, and that the field should therefore be revised to conform to
first principles determined by philosophy, would be considered to be working within the philosophy-first
paradigm. Many philosophers (if not most), and certainly most mathematicians, reject this paradigm. For the
purposes of this paper, I also reject this view and instead hope to provide a coherent, post-hoc account
of the aesthetic experience of mathematicians. 

One interesting consequence of this approach is that empirical observations of mathematicians' aesthetic experiences
become not only relevant to the discussion, but in many cases, necessary. This is especially so when considering the 
nature of the mathematical aesthetic, and how it relates to aesthetic experiences in other domains. Indeed, if we accept the
Wittgensteinian perspective on semantics, we must inquire into how the aesthetic lexicon is actually used by mathematicians if we hope to 
understand the meaning that they are trying to convey. As a result, much of the essential empirical research on this topic comes not only 
from those interested in mathematics and aesthetics, but also mathematics education \citep{dreyfus_aesthetics_1986,
inglis_beauty_2015,larvor_diversity_2016,inglis_are_2020,sa_mathematicians_2023}. I will refer to several of these 
important papers throughout this essay.

\section{Types of Aesthetic Judgments}
Although it is tempting to treat the aesthetic experience of mathematics as a monolithic phenomenon, mathematicians in
fact make several types of aesthetic judgments in various different contexts. 
\cite{sinclair_roles_2004,sinclair_aesthetic_2011} characterizes three different roles that aesthetic judgments play in mathematics:
\textit{evaluative}, \textit{generative}, and \textit{motivational}. Evaluative judgments are those that concern
the significance or elegance of particular mathematical results. These are the types of judgments made by publishers and
reviewers of academic journals in deeming whether or not particular results or theorems are worthy of 
publication, and the judgments according to which mathematicians determine how best to present their work for 
others to read. Evaluative judgments are perhaps the most easily recognizable
of the different types of judgments, as they are the ones for which aesthetic language is most often used and discussed.
Generative judgments, on the otherhand, are often made implicitly by mathematicians in the process of developing theories
and attempting to prove new results---they concern which strategies should be used and which avenues should be pursued
to most effectively solve a problem or provide a proof. Finally, motivational judgments are those that determine which
problems are most worthy of study in the first place and which areas of mathematics should be explored.

As \cite{ivanova_aesthetic_2017} points out, there is a similar feel to the distinction between these types of judgments 
and the distinction drawn by Reichenbach \citep{sep-reichenbach} between the \textit{context of discovery} and the
\textit{context of justification} in the philosophy of science. In the traditional interpretation of these two contexts,
the former occurs when a mathematician or scientist is exploring the space of possible explanations and theories (i.e., 
making motivational judgments), while the latter occurs when they are attempting to justify the truth of a particular 
theory or explanation (i.e., making generative judgments while developing a proof). Although the context of justification 
is generally considered to be a strictly rational process (and in mathematics, logical consistency certainly reigns above all other
considerations), \cite{cellucci_mathematical_2015} argues that aesthetic generative judgments also play a role 
in this context. Even this process, however, primarily entails selecting from among possible options; the differences is
that the selection occurs on the level of possible lemmas, methods of computation, and other aspects of detailed 
argumentation rather than occuring on the level of theory as a whole.

There now seems to be a general consensus that the types of judgments which we have traditionally deemed `aesthetic' judgments,
and which use a subset of the same aesthetic vocabulary with which we are familiar, truly do occur in both the context
of discovery and the context of justification, as well as the evaluative judgments made after-the-fact. The question of 
whether or not these judgments are truly aesthetic in nature is a separate one, and one which I will address in the next
section. 

% This threefold distinction provides a valuable framework for understanding the role of aesthetics in mathematics,
% and I will refer back to it throughout this paper, but it is certainly not complete. For example, it is unclear
% whether or not 

% The questions of whether or not
% the aesthetic qualities of these judgments are inherent to their objects or projected onto them by observers 
% (as argued by \cite{rota_phenomenology_1997}) remain unresolved, and 

\section{Is the Mathematical Aesthetic Really Aesthetic?}

The concept of the aesthetic has always been notoriously difficult to define, and has been linked to the concept
of taste since the days of Hume. For many early theorists, aesthetic
judgments were taken to be defined by immediacy and disinterest. By being immediate, it was meant that judgments
of beauty were not based on reason or rational deliberation, but were instead as direct and instinctive as
sense perception. By being disinterested, it was meant that aesthetic judgments were not based on any particular
desire or interest, but were instead made for their own sake. Hume in particular argued that although aesthetic
judgments, though immediate, were based on general principles about the world, an improved capacity for making
these judgments (also known as taste) could be cultivated through experience \citep{sep-aesthetic-concept}. 
This simplified picture of the early aesthetic tradition is, of course, much more complex and controversial 
in reality, but it does provide some important background for discussion in this paper.

In the realm of mathematics, the Humean position would be that as people develop expertise in the field, 
their capacity for making aesthetic judgments about mathematical objects and arguments improves 
because they develop a more refined mathematical taste. Indeed, this does appear to reflect reality in that
most mathematicians relate to perceiving a result as elegant or beautiful, while nine out of twelve of the
non-mathematicians in a survey conducted by \cite{zeki_experience_2014} indicated that they do not
``experience an emotional response'' when they encounter ``beautiful equations''. 

Another difficult question in aesthetics is whether or not aesthetic judgments are objective or subjective.
Objective aesthetic judgments are those that hold universally, independent of the observer, while subjective
aesthetic judgments are those that may vary from observer to observer because at their root, they are projected
onto their objects by the subject. A notion of inter-subjective agreement
is also often invoked, in which aesthetic judgments tend to be shared by a particular group or society. It is
with this conception of inter-subjective agreement that \cite{mcallister_mathematical_2005} put forth one
of the strongest explanations of the relationship between aesthetic and epistemic value in mathematics. He
presents a theory of ``aesthetic induction'', in which the aesthetic value that the mathematical (or scientific)
community place on a particular result or theory is a result of the state of the community's current ``aesthetic
canon'', and the degree to which the result accords with that canon. For example, the results of denying the 
parallel postulate were thought to be ugly and ``repugnant'' by the mathematical community up until the 19th 
century, when such non-Euclidean geometries were shown to be consistent models of the other axioms of Euclidean
geometry. Since that time, the aesthetic canon of the mathematical community has shifted to deemphasize graphical
representations and visual intuition, and to instead emphasize logical coherence and consistency. This shift
in the aesthetic canon has resulted in the acceptance of non-Euclidean geometries as very beautiful and elegant.

McAllister actually spends more time discussing the ``aesthetic induction'' in scientific theories than 
in mathematical theories (giving quantum mechanics as a particular case study), but the analogy holds up quite well
regardless. In essence, he argues that the aesthetic canon of a society
at a given time is ultimately determined by the empirical success of existing theories, and the properties such
theories are known to have. For example, since symmetry and parsimony are known to be properties of many 
successful theories, they are considered to be beautiful---but only for as long as they continue to be consistently 
associated with successful theories. This theory holds up well in the case of one recent development in 
mathematics---the rise of the computer-assisted proof, particularly to prove the four-color theorem. Since at no
time in history had proofs of mathematical theorems been provided which could not, at least in theory, be checked 
by holding the entire argument in one's head, the idea of a proof that could not be verified by a human was
disconcerting to many mathematicians. Indeed, \cite{rota_phenomenology_1997} explains that
\begin{quotation}
      Mathematicians have been ambivalent about such a verification. On the one hand, every mathematician 
      professes to be satisfied to learn that the conjecture has been settled. On the other hand, the 
      behavior of the community of mathematicians belies such a feeling of satisfaction.
\end{quotation}
This is an excellent example of the aesthetic canon of the mathematical community being challenged by a new
development in the field. The computer-assisted proof of the four-color theorem was not, and continues to not be 
accepted as beautiful by the community, but McAllister predicts that computer-assisted proofs can become more 
beautiful overtime as the standards of the community change due to the inter-subjective nature of the 
aesthetic \citep{mcallister_mathematical_2005}.

However, McAllister's position leads to some problems of its own, including the possibility that the aesthetic
judgments of mathematicians are really just epistemic judgments in disguise. \cite{todd_unmasking_2008} is perhaps
the most prominent example of someone sympathetic to this view. While he does not fully take the position  of 
aesthetic-epistemic identity himself, he provides strong grounds for suspecting that the distinction between the 
aesthetic and epistemic in mathematics is, in fact, illusory, and therefore argues that we should assume this to be 
the case until we can demonstrate otherwise. He first  claims that the vocabulary used to
describe mathematical beauty is much weaker than that used to define beauty in other domains. A scientist, engineer,
or data analyst may exclaim ``Beatiful!'' upon making a discovery or successfully processing a dataset, but we generally
interpret this exclamation as expressing only excitement, intellectual pleasure, or satisfaction rather than appreciation
of beauty. The argument is that a similar situation occurs in mathematics, and that the paucity of expressive power held
by the aesthetic vocabulary in mathematics is due to the fact that, at their core, the words used are not even expressing 
aesthetic judgments, but some other type of judgment. The second claim is that, when forced to explain their evaluation 
of an argument as aesthetically valuable, the criteria that mathematicians fall back on are almost always epistemically 
valuable as well. This is precisely the claim that \cite{mcallister_mathematical_2005} makes, and what lends credibility
to the theory of aesthetic induction. If all of the elements of \textit{elegant} proofs happen to coincide with those of 
\textit{valid} proofs, what evidence do we have to say that there is a distinction between the two types of value?

I find the first claim rather unconvincing. While it is true that the range of adjectives used by mathematicians to describe
the aesthetic qualities of their work is limited, this is not by itself evidence that they are not having an affective
experience. Indeed, in comparison to the size and scope of the art world (which forms the obvious basis of comparison),
the mathematical community is both much smaller and much less accessible to outsiders. Mathematicians constantly create 
and define new technical terms as they develop new mathematical structures and theories, but while most of humanity has,
at some point, taken time to appreciate paintings, music, and literature, only a small fraction of the population has 
taken the opportunity to appreciate mathematical proofs, and it is therefore not surprising that the vocabulary used 
to describe the aesthetic experience of mathematics is less fully developed than in the arts.

The second claim is more nuanced and is worth considering in more detail. While it is possible that a more thorough 
empirical investigation into the reasons mathematicians give for describing proofs as beautiful could reveal instances
where the criteria for aesthetic and epistemic value diverge, I do not here object to the current empirical claim 
that the criteria often coincide. Instead, I would like to further investigate the implications of this coincidence
to determine whether or not this precludes a distinction between the two types of value. In particular, I
would like to consider whether or not the putative conflation of the aesthetic and epistemic occurs in the same
way for evaluative, generative, and motivational judgments, or if there are differences between the three types.

\section{The Aesthetic-Epistemic Dichotomy}

Consider the experience of the mathematician in the context of discovery. In deciding what types of research questions
to ask and what types of problems to work on, the mathematician is guided by a number of factors, including the
perceived difficulty of the problem, the perceived importance of the problem, and the perceived likelihood of success.
While there are certainly many other complex and nuanced factors, I argue that epistemic values alone cannot guide
these decisions. For mathematicians and logicians, epistemic properties are binary---a proof is either valid or
invalid, a statement is either true or false, a theory is either consistent or inconsistent, and a proposition
is either provable or unprovable. When working within algebraic topology, analytic number
theory, or any other subfield, the mathematician has infinitely many possible questions to ask and problems to work
on, and the vast majority of them are either trivially true or trivially false \citep[p. 324-325]{poincare_mathematical_1910}. 
These propositions are not interesting,
although it doesn't appear that they lack any more epistemic value than deeper propositions. To preference one 
above another and actually make a decision, the mathematician must appeal to some other type of value, and I argue
that in the absence of evidence to the contrary, it is reasonable to assume that this value is aesthetic. Hardy
describes \textit{important} problems have the essentially aesthetic property of being \textit{serious}, and says 
that ``the seriousness of a mathematical 
theorem lies, not in its practical consequences, which are usually negligible, but in the significance of the 
mathematical ideas which it connects'' \citep[p. 16]{hardy_mathematicians_1940}. \cite{ivanova_aesthetic_2017} 
also claims that ``aesthetic values often drive the preference of one theory over another in the case of 
underdetermination''. Even if this process of selecting problems to work on is not guided by aesthetic value, 
I argue that the truth-functional nature of epistemic value makes it insufficient to explain the process.

Once the mathematician has chosen a problem to work on, she must then decide what approach to take to provide
a proof or solution. She is again faced with infinitely many possible approaches, but I can admit
here that in practice, only a vanishingly small number of these is likely to be successful, and in truth
only a finite number of these is even considered. \cite{poincare_mathematical_1910} discusses this generative 
process in greater detail, comparing the way in which the unconscious mind of an experienced mathematician does not
bother presenting the conscious mind with large numbers of false leads to the way in which instructors of graduate
courses do not bother assessing principles that were learned in high school.  A natural filtering process occurs
which depends on the mathematician's prior experience and knowledge of the subject. With concrete examples, 
Poincaré and other mathematicians have described this process as being guided chiefly by aesthetic factors, and to be 
emotionally affected by the intermediate results that are discovered along the way \citep{sinclair_roles_2004}.

One particularly interesting example comes from \citet{papert_mathematical_1978}, who analyzed whether or not
these aesthetic generative judgments could also be experienced by non-mathematicians. They described that when
given minimal instruction and guidance on how to prove that $\sqrt{2}$ is irrational, and only told to begin with
the initial equation $\sqrt{2}=p/q$, the vast majority of participants became excited and confident when the reached 
the form $p^2=2q^2$, despite understanding what next steps in the proof would need to be, or even understanding
the idea of a proof by contradiction in general. Papert claimed that the pleasure experienced by the participants
indicated that an aesthetic judgment had been made when they judged that form of the equation to be insightful, since 
the participants had no other way of evaluating their progress from an epistemic or utilitarian perspective.

Finally, once the mathematician has found a potential proof (i.e., has entered the context of justification), 
she must evaluate it. Ultimately, all relevant epistemic
considerations are subsumed in the question of whether or not the term ``proof'' is a valid description for the set
of symbols and manipulations that have been written down. Those symbols constitute a proof if and only if they
are epistemically and logically sound. Moreover, the truth-functional nature of logic means that one proof
of a theorem is exactly as valid as any other. And yet for some reason, mathematicians continue to prefer certain
proofs of results over others, and to search for proofs that are more elegant, more insightful, or more beautiful.
Why go about proving the same theorem in a different way, if an aesthetic judgment is not being made?

In tribute to what Paul Erdős called ``The Book'' (a hypothetical book in which God had written the most beautiful proofs
of mathematical theorems), mathematicians Martin Aigner and Günter Ziegler recently published a volume titled \textit{Proofs 
from THE BOOK},
which contains a collection of proofs that are particularly elegant, insightful, or beautiful. The book does not contain
any novel theorems, definitions, or results; everything within was previously known to the world. It also is not of 
particular value
from a pedagogical perspective; without an understanding of basic calculus, linear algebra, and number theory, a reader 
will likely not be able to understand or appreciate many of the proofs, and with a strong understanding of these topics,
the book will not provide much new insight. The book is not a reference text, and it is not a textbook. The book instead
claims to be a collection of proofs that are beautiful, offered only for readers' enjoyment. If the book does not
have aesthetic value as judged from an evaluative perspective, then it likely has no value at all \citep{aigner1999proofs}.

In summary, the mathematician claims that many factors affecting their work are aesthetic in nature, but \cite{todd_unmasking_2008} 
contends that we should be suspicious of this claim and that we should not simply take their statements at face value. 
Given that at each of the levels discussed above, something beyond the epistemic seems to be in play, where does this 
leave us now?

\section{Conjunctive and Disjunctive Perspectives}

Given the previous analysis, it does not seem possible to reduce the aesthetic judgments of mathematicians to
an epistemic basis, which seems to address Todd's concerns about the potential for aesthetic induction
theory. However, the original problems which motivated his initial inquiry have not, at their heart, been 
resolved \citep{todd_unmasking_2008}. Are the aesthetic and epistemic really
connected on a fundamental level, even if the aesthetic is not completely reducible to the epistemic? Or is there
a fundamental distinction between the two types of value?

\cite{todd_unmasking_2008} differentiates between the conjunctive position, in which there is a fundamental dependency
relation between beauty and truth, and the disjunctive position, in which there is not. 
In the disjunctive view, there should be 
instances of beautiful-and-true theorems, beautiful-yet-false theorems, ugly-yet-true theorems, and ugly-and-false theorems,
while in the conjunctive view, one of these categories should be empty. 
Mathematicians can provide plenty of examples for the beautiful-and-true \citep{aigner1999proofs} and
ugly-and-false theorems, and as noted earlier, the existence of the ugly-yet-true is what motivates mathematicians
to search for more elegant arguments and alternative proofs to results that have already been proven. But what about
the beautiful-yet-false? Are there any examples of arguments, conjectures, or other theories that were widely considered 
to be beautiful, but which were later shown to be false?

In the absence of valid examples, the disjunctive view is not necessarily disproven, but it is certainly weakened---as
it is reasonable to assume that the aesthetic and epistemic are at least somewhat connected, given how many 
mathematicians extol beauty as evidence of truth. I would like to present two possible examples of beautiful-yet-false
concepts in mathematics.
I find both of these examples to be compelling, but I have not yet conducted any empirical research into the views
of other mathematicians on these topics. While such research would be essential under an objective or inter-subjective
view of beauty, even under a subjective perspective, it would be useful to know whether or not other mathematicians
find these examples to be beautiful in order to determine whether my opinion is, in this case, idiosyncratic.

The first example is the Hilbert Program \citep{sep-hilbert-program}. The Hilbert Program was a project proposed by 
David Hilbert in the early 20th centruy to prove the consistency of arithmetic---and therefore provide a solid 
logical foundation for all of mathematics---by reducing 
arithmetic to a finite set of axioms and rules of inference. The program was widely regarded and well-known within
the mathematics and logic communities, and because of the beautiful idea that all of mathematics could be proven
to be consistent, it was widely believed to be true (yet unproven) until Kurt Gödel published his incompleteness
theorems in 1931, demonstrating that the program was impossible to complete.

The second example is the conjecture, believed by many mathematicians for thousands of years, that Euclid's fifth
postulate (also known as the \textit{parallel postulate}), given at the beginning of his \textit{Elements}, could 
be proven from the other four. This postulate states that given a line $\ell_1$ and a point $P$ not on that line, 
there is exactly one line $\ell_2$ through $P$ that is parallel to $\ell_1$. Euclid's other four postulates are all 
relatively simple, 
and were thus accepted unproblematically by mathematicians as axioms for geometry. Since the parallel
postulate was so unintuitive by comparison, geometers believed that its seeming necessity was ugly and inelegant, and
that its truth should be proven rather than assumed. This belief was so widespread that it was not until
the 19th century that mathematicians began to seriously consider the possibility that the parallel postulate was
truly independent, resulting in counterexamples to its necessity and the birth of the consistent mathematical
subfields of non-Euclidean geometry \citep{sep-epistemology-geometry}. 

Interestingly, I would also argue that both Gödel's incompletness theorems and non-Euclidean geometry are beautiful
as well. While truth must satisfy the principle of non-contradiction for epistemic claims to have meaning, the 
same is not generally true for beauty, and this is, I believe, a reason for which the beautiful-yet-false is possible.
The apparent existence of beautiful-yet-false concepts in mathematics lends further credence to the disjunctive
view of the relationship between the aesthetic and the epistemic. But it makes the apparent relationship between
beauty and truth, felt by many mathematicians, even more mysterious.

\section{Conclusion}

I have no particularly compelling explanation as for why truth seems to be so closely related to beauty in mathematics,
nor do I understand the nature of the mathematical sublime itself. The claims that mathematicians and scientists make
in trusting their aesthetic judgments to lead to epistemic truth seem unjustified, and the question of whether
aesthetic properties are essential to mathematical objects and arguments, or projected onto them by mathematicians,
remains an open one. Nevertheless, many of these questions about the nature of the mathematical aesthetic are
common to aesthetics in general. While further investigation into the relationship between truth and beauty is
necessary, the aesthetic and epistemic are not reducible to one another, and the nature of the mathematical
aesthetic must be a disjunctive one.

% \subsection{Beautiful and False}

% \begin{itemize}
%       \item The Beauty of the Langlands Program?? (actually this would be a great example for motivational, but not
%             evaluative)
%       \item The Beauty of the Hilbert Programme
%       \item The Beauty 
% \end{itemize}

% While the results of \cite{inglis_beauty_2015} are fascinating, it is important to note that their empirical
% semantics do not compare the way in which mathematicians use aesthetic language to the way in which non-mathematicians
% use aesthetic language. As a result, while they seem to have shown that individual mathematicians do use aesthetic
% language to describe mathematical objects and procedures in consistent ways---in particular, that there are
% at least four different dimensions along which they classify these mathematical structures---, they have not shown that this use
% of language is either consistent or inconsistent with the use of aesthetic discourse in other domains. 

% I want to argue that a disjunctive view of the relationship between aesthetics and epistemics in mathematics
% and science is entirely plausible, and provides a sufficient explanation for the use of aesthetic language
% in mathematics. This contradicts the view espoused by \cite{todd_unmasking_2008}, who argues that the
% use of aesthetic language in mathematics is entirely epistemic in nature. As particular counterexamples,
% I will use the instances in science of past models which were, and could still now be, considered beautiful (such
% as the Ptolemaic model of the solar system or the notion of a pervasive, light-conducting ether permeating the
% universe) and within mathematics, the entire Hilbert programme as an instance of a beautiful, but false theory.
% Since we also have plenty of examples of ``ugly'' truths in both mathematics and science, this seems mostly sufficient
% to demonstrate the validity of the disjunctive model of aesthetic experience in mathematics and science.

% However, I'm pretty sure there have to be some sort of flaws in that logic, since it seems so simple and easy. I know
% I can also draw on the work from \cite{inglis_beauty_2015} to support my conclusions, but I think there are deeper
% reasons for which \cite{todd_unmasking_2008} rejects the disjunctive model. I think I need to address those reasons,
% but I'm not currently in a sufficiently awake state to do so. I'll try to get back to this tomorrow.



\bibliography{example}

\end{document}

%
% Please see the package documentation for more information
% on the APA6 document class:
%
% http://www.ctan.org/pkg/apa6
%