\documentclass[a4paper,man,natbib]{apa6}

\usepackage[english]{babel}
% \usepackage[utf8x]{inputenc}
\usepackage{amsmath}
\usepackage{graphicx}
\usepackage[colorinlistoftodos]{todonotes}

\title{The Development of Taste and the Mathematical Aesthetic}
\shorttitle{Taste and the Mathematical Aesthetic}
\author{Braden Webb}
\affiliation{Department of Philosophy, Brigham Young University\\
PHIL 495R: Directed Readings in the Philosophy of Mathematics\\
Dr. Derek Haderlie\\
22 April 2023}

\abstract{Although mathematicians have long used aesthetic language to describe their work,
characterizing theorems and proofs as ``beautiful'' or ``elegant'', it is only in recent decades 
that a substantial literature on mathematical aesthetics has emerged in the philosophical community. 
In this paper, I argue for a disjunctive view of the relationship between aesthetics and epistemics in
mathematics.}

\begin{document}
\maketitle

\section{Introduction}
Carl Friedrich Gauss described mathematics as the ``queen of the sciences'', and its apparent purity and 
seemingly \textit{a priori} nature has led many to consider it the epitome of what can be achieved through
the application of logic and human reason. Because of these uniquely rigorous and epistemically sound
foundations, the field has often been upheld as a valuable case study for philosophers \citep{shapiro_thinking_2000},
demanding explanation for how knowledge of abstract mathematical objects and structures is possible. 
Yet despite the general emphasis on the epistemic properties of mathematics, working mathematicians have 
also been long known to use aesthetic language in describing their work, characterizing theorems and proofs as
``beautiful'' or ``elegant''. 

For example, \cite{hardy_mathematicians_1992} writes that
\begin{quotation}
      The mathematician’s patterns, like the painter’s or the poet’s, must be beautiful; the ideas, like the 
      colours or the words, must fit together in a harmonious way. Beauty is the first test: there is no 
      permanent place in the world for ugly mathematics.
\end{quotation}
He also argues that the aesthetic value of a mathematical theorem depends on many properties, including
beauty, seriousness, generality, depth, unexpectedness, inevitability, and 
economy \citep{hardy_mathematicians_1992}. A recent empirical survey \citep{johnson_intuitions_2019} finds that mathematicians
tend to consistently describe the aesthetic properties of mathematical arguments along dimensions of
elegance, profundity, and clarity. Indeed, Aristotle himself claimed that ``the main species of beauty are orderly 
arrangement, proportion, and definiteness'', and that ``these are especially manifested by the mathematical sciences''
\cite[VIII, 1078a]{aristotle_metaphysics_nodate}.

In recent decades, a substantial literature on the mathematical aesthetic has emerged. 
\cite{sinclair_roles_2004} characterizes three different roles that aesthetic judgments play in mathematics:
\textit{evaluative}, \textit{generative}, and \textit{motivational}. Evaluative judgments are those which


In considering the study of the philosophy of mathematics, \cite{shapiro_thinking_2000} characterizes
the view that philosophy ``precedes'' or ``determines'' practice as the \textit{philosophy-first} paradigm.
For example, those who subscribe to the view that modern mathematical discourse is incoherent due to unsound
metaphysical or logical presuppositions, and that the field should therefore be revised to conform to
first principles determined by philosophy, would be considered to be working within the philosophy-first
paradigm. Many (if not most) philosophers, and certainly most mathematicians, reject this paradigm. For the
purposes of this paper, I also reject this view and instead hope to provide a coherent, post-hoc account
of the aesthetic experience of mathematicians. One interesting consequence of this approach is that empirical
observations of mathematicians' aesthetic experiences become relevant to the discussion, and the surveys
conducted by \cite{inglis_beauty_2015} of contemporary research mathematicians will be used to support several
of my arguments. 

\begin{itemize}
      \item What is taste? - Hume, Bordieu, Kant...
      \item How is taste acquired? (Maybe mention \cite{mcallister_mathematical_2005}?)
      \item The development of taste in evaluative judgments
      \item The development of taste in generative judgments (Thomas argues that 
      interest is more important than beauty in these judgments, along with 
      Poincare and Hadamard as referenced in Sinclair)
      \item The development of taste in motivational judgments
      \item The distinction between ``interesting'' and ``important'', as indicated
      both by Hardy and Thomas
      \item Conjunctive vs disjunctive accounts of aesthetic judgment
\end{itemize}





While the results of \cite{inglis_beauty_2015} are fascinating, it is important to note that their empirical
semantics do not compare the way in which mathematicians use aesthetic language to the way in which non-mathematicians
use aesthetic language. As a result, while they seem to have shown that individual mathematicians do use aesthetic
language to describe mathematical objects and procedures in consistent ways---in particular, that there are
at least four different dimensions along which they classify these mathematical structures---, they have not shown that this use
of language is either consistent or inconsistent with the use of aesthetic language in other domains. 


I want to argue that a disjunctive view of the relationship between aesthetics and epistemics in mathematics
and science is entirely plausible, and provides a sufficient explanation for the use of aesthetic language
in mathematics. This contradicts the view espoused by \cite{todd_unmasking_2008}, who argues that the
use of aesthetic language in mathematics is entirely epistemic in nature. As particular counterexamples,
I will use the instances in science of past models which were, and could still now be, considered beautiful (such
as the Ptolemaic model of the solar system or the notion of a pervasive, light-conducting ether permeating the
universe) and within mathematics, the entire Hilbert programme as an instance of a beautiful, but false theory.
Since we also have plenty of examples of ``ugly'' truths in both mathematics and science, this seems mostly sufficient
to demonstrate the validity of the disjunctive model of aesthetic experience in mathematics and science.

However, I'm pretty sure there have to be some sort of flaws in that logic, since it seems so simple and easy. I know
I can also draw on the work from \cite{inglis_beauty_2015} to support my conclusions, but I think there are deeper
reasons for which \cite{todd_unmasking_2008} rejects the disjunctive model. I think I need to address those reasons,
but I'm not currently in a sufficiently awake state to do so. I'll try to get back to this tomorrow.

\bibliography{example}

\end{document}

%
% Please see the package documentation for more information
% on the APA6 document class:
%
% http://www.ctan.org/pkg/apa6
%